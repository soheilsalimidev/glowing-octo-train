\section{Data Cleaning And Feature Extraction}
All files were resampled to 16~kHz, its because can captures the full
range of human phonetic
content (up to 8~kHz) \cite{wiki:Voice_frequency}. This reduce
computational needed compared
to higher rates like 44.1~kHz.

Each recording was truncated
or zero-padded to exactly 60 seconds to ensure uniform input length,
this make sure  feature dimensions to be consistent. We extracted
Mel-Frequency Cepstral
Coefficients (MFCCs) because
they model the human auditory system’s perception of sound and
capture language-specific phonetic.
For each MFCC dimension, we computed its mean and SD across time,
making a compact 26-dimensional feature
vector that has both the average spectral shape and its
variability. We
normalize all features to zero mean and unit variance. This step is
needed for distance-based algorithms like SVM, to
prevent features with larger numerical ranges from disproportionately
influencing similarity measures.
